\documentclass{homework}
\usepackage{homework}
\usepackage{hanhua}
\title{Week 8}
\date{}
\begin{document}
\maketitle
\section{(P136/1)}
说明极值原理中$c(x)$为负时,极值原理可能不成立。考虑一维的Helmholtz方程,找$u$满足$$\frac{\diff ^2u}{\diff x^2}+k^2u=1,~u(0)=u(1)=0.$$
\section{(P136/3)}
设$u(0)=u(1)=0$且满足$-\frac{\diff ^2u}{\diff x^2}=f(x).$利用Green函数表达式
\begin{eqnarray*}
G(x)=
\begin{dcases}
(1-x_0)x, &0<x<x_0,\cr x_0(1-x), &x_0<x<1.
\end{dcases}
\end{eqnarray*}
证明$$u(x)=\int_0^1G(x;x_0)f(x_0)\diff x_0.$$
\section{解:}
$$-\frac{\diff ^2u}{\diff x^2}=\delta(x-x_0),~G(0)=0,~G'(1)=0.$$
\section{(P143/3改)}
对方程$$-a\frac{\diff ^2u}{\diff x^2}+b\frac{\diff u}{\diff x}+cu=1,$$如果边界条件改为$u(1)=0,\frac{\diff u}{\diff x}(0)+u(0)=0,$求其精确解;并用三点差分格式离散求解,同时分析差分格式的精度。
\section{用三点差分格式解:}
\begin{eqnarray*}
\begin{dcases}
-\frac{\diff^2u}{\diff x^2}+u=f,&~x\in(0,1)\cr u(0)=\alpha,~u(1)=\beta.
\end{dcases}
\end{eqnarray*}
其中$f,\alpha,\beta$由$u=\cos x$得到。即$f=2\cos x,~u(0)=1,~u(1)=\cos 1.$
\section{(P151/2)}
证明不等式$\bignorm{\bf e}_{\ell^2}\leqslant\bignorm{\delta^+_x{\bf e}}_{\ell^2},$并用它来证明当$c(x)=0$时,下述方程三点差分格式的收敛性:$$-\frac{\diff^2u}{\diff x^2}+u=f(x),~u(0)=u(1)=0.$$且三点差分格式函数值和导数值都是两阶收敛的。
\end{document}
